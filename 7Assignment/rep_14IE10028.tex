\documentclass{article}
\usepackage[utf8]{inputenc}

\usepackage{listings}
\usepackage{color}

%New colors defined below
\definecolor{codegreen}{rgb}{0,0.6,0}
\definecolor{codegray}{rgb}{0.5,0.5,0.5}
\definecolor{codepurple}{rgb}{0.58,0,0.82}
\definecolor{backcolour}{rgb}{0.95,0.95,0.92}

%Code listing style named "mystyle"
\lstdefinestyle{mystyle}{
  backgroundcolor=\color{backcolour},   commentstyle=\color{codegreen},
  keywordstyle=\color{magenta},
  numberstyle=\tiny\color{codegray},
  stringstyle=\color{codepurple},
  basicstyle=\footnotesize,
  breakatwhitespace=false,         
  breaklines=true,                 
  captionpos=b,                    
  keepspaces=true,                 
  numbers=left,                    
  numbersep=5pt,                  
  showspaces=false,                
  showstringspaces=false,
  showtabs=false,                  
  tabsize=2
}

%"mystyle" code listing set
\lstset{style=mystyle}

\title{Red Black Tree}
\author{Shashwat Gupta (14IE10028)}

\begin{document}

\maketitle

\section{Insertion}

The following is the code for insertion of a node into the Red Black tree. The code is similar to insertion in a binary search tree. The insertion code is followed by a insertion fix code which does the required rotations and colour changing to balance and meet the conditions of the red black tree.

The following is the Code for insert
\begin{lstlisting}[language=c, caption=insert]
node* insert(node* z, node* root)
{
    node *x = (node*)malloc(sizeof(node));
    node *y = (node*)malloc(sizeof(node));
    y=root;
    x=root;
    while(x->extime<=1000)
    {
        y=x;
        if(z->extime<x->extime)
            x=x->lChild;
        else
            x=x->rChild;
    }
    z->parent=y;
    if(y->extime>1000)
        root=z;
    else if (z->extime<y->extime)
        y->lChild=z;
    else
        y->rChild=z;
    z->lChild=&nil;
    z->rChild=&nil;
    z->colour=RED;
    root=insertFix(z,root);
    return root;
}
\end{lstlisting}

The following is the Code for insertfix
\begin{lstlisting}[language=c, caption=insert fix]

\end{lstlisting}

\section{Rotations}

The following are the codes for left and right rotations respectively.
These codes are responsible for the changes in the pointer structure as well as the changes in colour.
The following is the Code
\begin{lstlisting}[language=c, caption=left]
node* leftRotate(node* x, node* root)
{
    node *y = (node*)malloc(sizeof(node));
    y=x->rChild;
    x->rChild=y->lChild;
    if(y->lChild->extime<=1000)
        y->lChild->parent=x;
    y->parent=x->parent;
    if(x->parent->extime>1000)
        root=y;
    else if(x==x->parent->lChild)
        x->parent->lChild=y;
    else
        x->parent->rChild=y;
    y->lChild=x;
    x->parent=y;
    return root;
}
\end{lstlisting}

The following is the Code
\begin{lstlisting}[language=c, caption=Right]
node* rightRotate(node* x, node* root)
{
    node *y = (node*)malloc(sizeof(node));
    y=x->lChild;
    x->lChild=y->rChild;
    if(y->rChild->extime<=1000)
        y->rChild->parent=x;
    y->parent=x->parent;
    if(x->parent->extime>1000)
        root=y;
    else if(x==x->parent->lChild)
        x->parent->lChild=y;
    else
        x->parent->rChild=y;
    y->rChild=x;
    x->parent=y;
    return root;
}
\end{lstlisting}

\section{Deletion}

The deletion Subroutine takes in the node to be deleted and the root of the tree and removes the node from the tree. The delete along with the delete fix functions are responsible for removing the node from the tree and the rotations and colour changing operations.

The following is the Code
\begin{lstlisting}[language=c, caption=delete]
node* delete(node* z, node* root)
{
    node *x, *y;
    if(z->lChild->extime>1000||z->rChild->extime>1000)
        y=z;
    else
        y=treeSuccessor(z);
    if(y->lChild->extime<=1000)
        x=y->lChild;
    else
        x=y->rChild;
    x->parent=y->parent;
    if(y->parent->extime>1000)
        root=x;
    else if(y==y->parent->lChild)
        y->parent->lChild=x;
    else
        y->parent->rChild=x;
    if(y!=z)
    {
        z->extime=y->extime;
        z->id=y->id;
        z->priority=y->priority;
        z->colour=y->colour;
    }
    if(y->colour==BLACK)
        deleteFix(x,root);
    return y;
}
\end{lstlisting}


The following is the Code
\begin{lstlisting}[language=c, caption=deleteFix]
node* deleteFix(node *x, node *root)
{
    node *w;
    while((x->extime<=1000)&&(x->colour==BLACK))
    {
        if(x==x->parent->lChild)
        {
            w=x->parent->rChild;
            if(w->colour==RED)
            {
                w->colour=BLACK;
                x->parent->colour=RED;
                root=leftRotate(x->parent,root);
                w=x->parent->rChild;
            }
            if( w->lChild->colour==BLACK&&w->rChild->colour==BLACK)
            {
                w->colour=RED;
                x=x->parent;
            }
            else
            {
                if(w->rChild->colour==BLACK)
                {
                    w->lChild->colour=BLACK;
                    w->colour=RED;
                    root=rightRotate(w,root);
                    w=x->parent->rChild;
                }
                w->colour=x->parent->colour;
                x->parent->colour=BLACK;
                w->rChild->colour=BLACK;
                root=leftRotate(x->parent,root);
                return x;
            }
        }
        else
        {
            w=x->parent->lChild;
            if(w->colour==RED)
            {
                w->colour=BLACK;
                x->parent->colour=RED;
                root=rightRotate(x->parent,root);
                w=x->parent->lChild;
            }
            if( w->rChild->colour==BLACK&&w->lChild->colour==BLACK)
            {
                w->colour=RED;
                x=x->parent;
            }
            else
            {
                if(w->lChild->colour==BLACK)
                {
                    w->rChild->colour=BLACK;
                    w->colour=RED;
                    root=leftRotate(w,root);
                    w=x->parent->lChild;
                }
                w->colour=x->parent->colour;
                x->parent->colour=BLACK;
                w->lChild->colour=BLACK;
                root=rightRotate(x->parent,root);
                return x;
            }
        }
    }
    return root;
}
\end{lstlisting}

\section{Process management}

The folowwing are the codes for process create and schedule which are responsible for the checking of N and creation of new nodes and processing inserted nodes.

The following is the Code
\begin{lstlisting}[language=c, caption=process create]
node* processCreate(int liveproc, node* root)
{
    node *new = (node*)malloc(sizeof(node));
    new->id=liveproc;
    new->extime=rand()%1000+1;
    new->priority=rand()%4+1;
    new->colour=RED;
    new->lChild=&nil;
    new->rChild=&nil;
    new->parent=&nil;
    root=insert(new,root);
    return root;
}
\end{lstlisting}

The following is the Code
\begin{lstlisting}[language=c, caption=process schedule]
void processSchedule(node *root)
{
    node *x;
    x=treeMin(root);
    delete(x,root);
    x->extime=x->extime-(50*x->priority);
    insert(x,root);
}
\end{lstlisting}

\section{Time Complexity Calculation}

Time complexity has the recursive relation:

\[ T(n) = T(n/2) + \theta(1) \]

where n is the number of nodes.

Solving this equation by recursive method yields T(n) $\in O(log n)$.

The equation holds for the Insertion aswell as the Deletion subroutine
\end{document}