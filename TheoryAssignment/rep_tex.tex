\documentclass{article}
\usepackage[utf8]{inputenc}

\usepackage{listings}
\usepackage{color}

%New colors defined below
\definecolor{codegreen}{rgb}{0,0.6,0}
\definecolor{codegray}{rgb}{0.5,0.5,0.5}
\definecolor{codepurple}{rgb}{0.58,0,0.82}
\definecolor{backcolour}{rgb}{0.95,0.95,0.92}

%Code listing style named "mystyle"
\lstdefinestyle{mystyle}{
  backgroundcolor=\color{backcolour},
  commentstyle=\color{codegreen},
  keywordstyle=\color{magenta},
  numberstyle=\tiny\color{codegray},
  stringstyle=\color{codepurple},
  basicstyle=\footnotesize,
  breakatwhitespace=false,         
  breaklines=true,                 
  captionpos=b,                    
  keepspaces=true,                 
  numbers=left,                    
  numbersep=5pt,                  
  showspaces=false,                
  showstringspaces=false,
  showtabs=false,                  
  tabsize=2
}

%"mystyle" code listing set
\lstset{style=mystyle}

\title{Dynamic Programming Assignment}
\author{Shashwat Gupta (14IE10028)}

\begin{document}

\maketitle

\section{Satisfying parenthesisation of Boolean expressions /w OR (Problem 7)}

The following is the code for my function which returns the number of parenthesis possible. 
I make use of two 2-D arrays and store the bottom-up table in them for true and false respectively. 
The tables are then evaluated using a formula for OR gate and the expression is checked for true or false and then updated accordingly.

The code
\begin{lstlisting}[language=c, caption=Parenthesis]
int countParenth(char expression[], int n)
{
    int i,j,g,gap;
    int F[n][n], T[n][n];
    for (i = 0; i < n; i++)
    {
        F[i][i] = (expression[i] == 'F')? 1: 0;
        T[i][i] = (expression[i] == 'T')? 1: 0;
    }
    for (gap=1; gap<n; ++gap)
    {
        for (i=0, j=gap; j<n; ++i, ++j)
        {
            T[i][j] = 0;
            F[i][j] = 0;
            for (g=0; g<gap; g++)
            {
                // Find place of parenthesization using current value of gap
                int k = i + g; 
                // Store Total[i][k] and Total[k+1][j]
                int tik = T[i][k] + F[i][k];
                int tkj = T[k+1][j] + F[k+1][j];
                // Follow the recursive formula
                F[i][j] += F[i][k]*F[k+1][j];
                T[i][j] += (tik*tkj - F[i][k]*F[k+1][j]);
            }
        }
    }
    return T[0][n-1];
}
\end{lstlisting}

\textbf{Complexity Analysis}

These are the costs: \\

1. Time Complexity  $\in O(n^3)$.

2. Space Complexity  $\in O(n^2)$.

\section{Knapsack packing with item repetition (Problem 11)}

We accept the details for all the items and the knapsack in hand. We then make use of an array K which store the maximum values for each capacity i in K[i]. thus, our objective is to obtain K[capacity];
For 0 capacity, max value = 0, thus K[0]=0
for all other i, first K[i] is initialised to K[i-1] as this is obvious and then we check with every element in hand along with the K of (total capacity - capacity of that element in hand). The maximum is chosen and K[i] is updated.
This is repeated for all n.

The following is the code
\begin{lstlisting}[language=c, caption=knapsack]
int main()
{
    int i,j,capacity,n;
    printf("Enter capacity of knapsack\n");
    scanf("%d",&capacity);
    printf("Enter number of items\n");
    scanf("%d",&n);
    int val[n];
    int size[n];
    for (i = 0; i < n; ++i)
    {
        printf("Value for item %d = ",i+1);
        scanf("%d",&val[i]);
        printf("Weight for item %d = ",i+1);
        scanf("%d",&size[i]);
    }
    int K[capacity+1];
    K[0] = 0;
    for (i = 1; i <= capacity; ++i)
    {
        K[i]=K[i-1];
        for (j = 0; j < n; ++j)
        {
            if((i-size[j])>=0 && K[i]<(K[i-size[j]]+val[j]))
                K[i]=K[i-size[j]]+val[j];
        }
    }
    printf("Maximum Value = %d\n", K[capacity]);
    return 0;
}
\end{lstlisting}

\textbf{Complexity Analysis}

These are the costs: \\

1. Time Complexity  $\in O(n*capacity)$.

2. Space Complexity  $\in O(capacity)$.

\end{document}