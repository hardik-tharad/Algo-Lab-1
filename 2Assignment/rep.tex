\documentclass[a4paper,11pt]{article}
\usepackage[utf8]{inputenc}
\usepackage{times}
\usepackage{amsmath}

\title{Report for assignment 2}
\author{Shashwat Gupta (14IE10028)}

\begin{document}

\maketitle

\paragraph{Fill a nxn board with one cell missing with L shaped tiles}
\begin{enumerate}
\item \textbf{Implementing structure}
The structure is implemented to store two integers x and y which are the coordinates on the board

\item \textbf{Implementing recursive function to split the board into 4 equal squares of size n/2}

This problem can be solved using Divide and Conquer. Below is the recursive algorithm.
Function \textbf{rec\_fill} is the recursive function doing the following things:
\begin{description}
 \item[$\bullet$] Find the Centre.
 \item[$\bullet$] Call function to find the quadrant that the missing/defective cell lies in.
 \item[$\bullet$] Call function to fill L tile appropriately.
\end{description}

\textbf{Base Condition for termination of recursion} \newline
\emph{	if(size==2)} \newline
\emph{	return;}

\textbf{Finding centre} \newline
\emph{	centre.x=start.x+size/2-1;}\newline
\emph{	centre.y=start.y+size/2-1;}

If the Base condition is not satisfied, the function first checks for which zone the defect is in and fills the L tile accordingly.
It then effectively divides the board into 4 equal squares each of size $\frac{n}{2}$, and implements recursion by calling each of the 4 sub-boards recursively.
Before calling each sub-board, it updates the start point and the defect point and passes them.

Sample code for calling the four sub-boards recursivley. This code is for when the defect is in zone 1.

\emph{	startCopyy.x=start.x;} \newline
\emph{	startCopyy.y=start.y;} \newline
\emph{	defectCopyy.x=defect.x;} \newline
\emph{	defectCopyy.y=defect.y;} \newline
\emph{	recfill(startCopyy,size/2,defectCopyy);} \newline
\emph{	startCopyy.x=start.x+size/2;} \newline
\emph{	startCopyy.y=start.y;} \newline
\emph{	defectCopyy.x=start.x+size/2;} \newline
\emph{	defectCopyy.y=start.y+size/2-1;} \newline
\emph{	recfill(startCopyy,size/2,defectCopyy);} \newline
\emph{	startCopyy.x=start.x+size/2;} \newline
\emph{	startCopyy.y=start.y+size/2;} \newline
\emph{	defectCopyy.x=start.x+size/2;} \newline
\emph{	defectCopyy.y=start.y+size/2;} \newline
\emph{	recfill(startCopyy,size/2,defectCopyy);} \newline
\emph{	startCopyy.x=start.x;} \newline
\emph{	startCopyy.y=start.y+size/2;} \newline
\emph{	defectCopyy.x=start.x+size/2-1;} \newline
\emph{	defectCopyy.y=start.y+size/2;} \newline
\emph{	recfill(startCopyy,size/2,defectCopyy);}

\item \textbf{Function that checks for quadrant} \newline
\emph{	int checkdefectzone(point centre, point defect)} \newline
\emph{		if(defect.x more than centre.x)} \newline
\emph{			if(defect.y more than centre.y)} \newline
\emph{				return 3;} \newline
\emph{			else} \newline
\emph{				return 2;} \newline
\emph{		else} \newline
\emph{			if(defect.y more than centre.y)} \newline
\emph{				return 4;} \newline
\emph{			else} \newline
\emph{				return 1;}

\item \textbf{Function that fills the L tile appropriately} \newline
\emph{	void fill4operation(point a, int z)} \newline
\emph{		if(z==1)} \newline
\emph{			arr[a.x+1][a.y+1]=printer;} \newline
\emph{			arr[a.x][a.y+1]=printer;} \newline
\emph{			arr[a.x+1][a.y]=printer;} \newline
\emph{		else if(z==2)} \newline
\emph{			arr[a.x][a.y]=printer;} \newline
\emph{			arr[a.x][a.y+1]=printer;} \newline
\emph{			arr[a.x+1][a.y+1]=printer;} \newline
\emph{		else if(z==3)} \newline
\emph{			arr[a.x][a.y]=printer;} \newline
\emph{			arr[a.x+1][a.y]=printer;} \newline
\emph{			arr[a.x][a.y+1]=printer;} \newline
\emph{		else} \newline
\emph{			arr[a.x][a.y]=printer;} \newline
\emph{			arr[a.x+1][a.y]=printer;} \newline
\emph{			arr[a.x+1][a.y+1]=printer;} \newline
\emph{		printer++;}


\item \textbf{Printing the Board} \newline
\emph{	void printarray(int size)} \newline
\emph{		printf("printing board: ");} \newline
\emph{		int i,j;} \newline
\emph{		for (i = 0; i less than size; ++i)} \newline
\emph{			for (j = 0; j less than size; ++j)} \newline
\emph{			printf("d\t",arr[i][j]);} \newline
\emph{			printf(" ");}

While printing the board, we observe a series of number in the board, where we observe all integers three times except the integer -1. The -1 represents the original missing cell and all the other integer triplets represent the L tile.

\item \textbf{Some Notes}
The baord has been assumed to be a 2-D array and its has been defined globally. Its size is user defined with an upper limit m which is macro defined and can be changed. A printer variable is also globally defined to store the running triplet number.

\item \textbf{Time Complexity Calculation}
For each value of size, the board checks if the size is 2X2 and if it isn't then it recursively calls the 4 sub-boards of size $\frac{n}{2}$

\begin{displaymath}   T(n) = \left\{
     \begin{array}{lr}
       a & : n = 1\\
       4T(\frac{n}{2}) + c & : n > 1
     \end{array}
   \right.
\end{displaymath}

Solving this equation by recursive method yields T(n) $\in O(n^2)$.
 
\end{enumerate}

\paragraph{Finding the closest pair of points}
\begin{enumerate}
\item \textbf{Implementing structure}
The structure is implemented to store two integers x and y which are the coordinates on the board

 \item \textbf{Implementing function to find the distance between 2 points}
 \newline
\emph{	float givedistance(point a, point b)} \newline
\emph{  return sqrt((a.x-b.x)*(a.x-b.x)+(a.y-b.y)*(a.y-b.y));}

\item \textbf{Implementing function to randomly populate the set of coordinate points with whole numbers less than 100}
\newline \emph{ void fillcoordinates() }
 \newline \emph{   srand(time(NULL)); }
 \newline \emph{   int i; }
 \newline \emph{   printf("Enter number of points : "); }
 \newline \emph{   scanf("d",n); }
 \newline \emph{   for (i = 0; i < n; ++i) }
 \newline \emph{       a[i].x=rand() mod 100; }
 \newline \emph{       a[i].y=rand() mod 100; }

 \item \textbf{Implementing function to print}
\newline \emph{ void printarray() }
 \newline \emph{   int i; }
 \newline \emph{   for (i = 0; i < n; ++i) }
 \newline \emph{       printf("(d,d) ",a[i].x,a[i].y); }
 \newline \emph{   printf(" "); }

\item \textbf{Implementing function to return least distance}
\newline \emph{float bruteforce()}
 \newline \emph{   int i,j;}
\newline \emph{    float min=givedistance(a[0],a[1]);}
\newline \emph{    for(i=0;i<n-1;i++)}
 \newline \emph{       for(j=i+1;j<n;j++)}
  \newline \emph{          if(givedistance(a[i],a[j])<min)}
 \newline \emph{               min=givedistance(a[i],a[j]);}
 \newline \emph{   return min; }

\item \textbf{Time Complexity Calculation}
For each value the board checks the distance with all the other values and finds the minimum

Solving this equation by recursive method yields T(n) $\in O(n^2)$.

\end{enumerate}

\end{document}