\documentclass[a4paper,11pt]{article}
\usepackage[utf8]{inputenc}
\usepackage{times}
\usepackage{amsmath}

\title{Report for assignment 1}
\author{Shashwat Gupta (14IE10028)}

\begin{document}

\maketitle

\paragraph{Multiplication of two big integers using FFT and IFFT}
\begin{enumerate}
 \item \textbf{Implementing FFT/IFFT}

FFT of a polynomial is found by evaluating the polynomial at $n^{th}$ roots of unity. We solve FFT recursively using divide and conquer by separating the odd and even coefficients, and obtain a time complexity of \emph{O(nlogn)}. We use Eulers formulas to compute the $n^{th}$ roots of unity. We can derive IFFT also in a similar way.

\item \textbf{Generation Random Big Integers}

Big integers are stored using data structures, dynamically allocated with random numbers.
Each number is represented as complex number of the form \textbf{a + ib} implemented using structures
A complex data structure is made which hold two contents: real part \emph{r} and imaginary part \emph{im}, both of which are of \emph{double} datatype.

At the runtime, user is asked to input number of digits, for two numbers and the big integer array is created using \emph{create} and it is first allocated space and then it is populated with random numbers (0-9 only), while keeping in mind that Most Significant Digit should not be zero.

\item \textbf{Multiplication algorithm}

Let integer A and B be represented as
\begin{equation}
A = \sum_{k=0}^{k=n-1} a_{k}x^k
\end{equation}
and 
\begin{equation}
B = \sum_{k=0}^{k=n-1} b_{k}x^k
\end{equation}

Let applying FFT for order 2n yield the respective fourier transforms as F(A) and F(B) defined by

\begin{equation}
F(A) = [A_{0}, A_{1}, A_{2}, ..., A_{2n-1}]\end{equation} where \begin{equation}
A_i = \sum_{k=0}^{k=2n-1} a_{k}w_{2n}^{ik}\end{equation}

\begin{equation}
F(B) = [B_{0}, B_{1}, B_{2}, ..., B_{2n-1}]\end{equation} where \begin{equation}
B_i = \sum_{k=0}^{k=2n-1} b_{k}w_{2n}^{ik}\end{equation}

where \begin{equation}
w_{2n}^k = e^(\frac{2.pi.k}{2n}) \end{equation}

\begin{equation}
a_{k} = b_{k} = 0 for k > n-1 \end{equation}

Let C be the product of A and B i.e. \textbf{C = AB}

Then fourier transform of C is given by

\begin{equation}
F(C) = [C_{0}, C_{1}, C_{2}, ..., C_{2n-1}] \end{equation} where \begin{equation}
C_i = A_i * B_i\end{equation}

The integer C is therefore obtained by taking IFFT of F(C).

\item \textbf{Time Complexity Calculation}

In FFT, for each value of n, two recursive calls of length $\frac{n}{2}$ is performed and rest of combining operation is performed in linear time therefore

 \begin{displaymath}   T(n) = \left\{
     \begin{array}{lr}
       a & : n = 1\\
       2T(\frac{n}{2}) + bn +c & : n > 1
     \end{array}
   \right.
\end{displaymath}

Solving this equation by recursive method yields T(n) $\in \Theta (n.lg(n))$.
 
\end{enumerate}

\end{document}
